\thispagestyle{empty}
{\theme{Biomechanical finite-element analysis of the lost jaw area restoration with an autograft.}}

\vspace{5mm}

\begin{center} Eduard B. Demishkevich$^{1,2}$, Sergey S. Gavriushin$^{1,2}$, Sergey S. Ivanov $^{1}$, Vladimir A. Perfiliev$^{3}$  \\

\vspace{5mm}

{\small $^{1}$ Bauman Moscow State Technical University, 5c1, 2nd Baumanskaya st., 105005, Moscow, Russian Federation \\
$^{2}$ Mechanical Engineering Research Institute of the Russian Academy of Sciences, 4, Malyi Kharitonievsky pereulok, 101990, Moscow, Russian Federation \\
$^{3}$  \\
mail@edtech.ru 
}
\end{center}

\begin{changemargin}{0.5cm}{0.5cm}
{\small \textbf{Abstract.} 
The study describes the methodology of patient-specific biomechanical analysis for surgery procedure of the lost jaw area 
restoration with an autograft. A three dimensional model of the patient's dentition was developed on the basis of layerwise image data obtained
using computer tomography. Geometrical characteristics of an autograft and anchorage scheme have been proposed by clinician during surgery planning.
A method to simulate impact of muscle activity on the biomechanical system has been presented in the study. 
The developed approach allows to make sure that the restored dentition suits strength requirements.
}

{\small \textbf{Keywords:} finite-elements method, biomechanics}

\end{changemargin}

\ctheme{1. Introduction}

% описать процедуру аутоимплантации

% рассказать о предыдущих работах, где использовалось КЭ моделирование в стоматологии

% научная новизна - предложена схема нагружения, моделирующая мышечное воздействие

% описать цель работы

\ctheme{2. Preparation of the patient-specific finite-element model}

% описать процесс создания модели

\ctheme{3. Loading scheme}

% описать как задавались граничные условия и нагрузки

\ctheme{4. Simulation results}

% показать полученные результаты моделирования

\ctheme{5. Discussion}

% провести анализ полученных результатов

\ctheme{6. Conclusion}

% описать что сделано

\ctheme{References}